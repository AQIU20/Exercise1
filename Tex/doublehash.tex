\nsection{Double Hashing}
Here we use double hashing, a technique for resolving collisions in a
hash table.

\begin{exercise}
\label{ex:dblhash}
Use double hashing to create a spelling checker, which reads in a dictionary file
from \verb^argv[1]^, and stores the words.

Make sure the program:
\begin{itemize}
\item Use double hashing to achieve this.
\item Makes no assumptions about the maximum size of the dictionary files. Choose
an initial (prime) array size, created via malloc(). If this gets more than $60\%$ full,
creates a new array, roughly twice the size (but still prime). Rehash all the words into this
new array from the old one. This may need to be done many times as more and more words
are added.
\item Uses a hash, and double hash, function of your choosing.
\item Once the hash table is built, reads another list of words from \verb^argv[2]^
and reports on the {\em average} number of  look-ups required. A {\em perfect} hash
will require exactly $1.0$ look-up. Assuming the program works correctly,
this number is the only output required from the program.
\end{itemize}
\end{exercise}
