\nsection{ANSI Escape Sequences}
\label{sec:ansi}
C has no inherent functionality to allow printing in colour etc.
Therefore, some terminals allow certain control codes to be printed
to move the cursor, change the foreground and background colour, and
so on.
Note that only some terminals support these control code - if you're
using a different one, garbage may appear on the screen instead.
\wwwurl{https://en.wikipedia.org/wiki/ANSI_escape_code}

I've created a very simple set of functions which shows a small
fraction of this functionality called \verb^demo_neillsimplescreen.c^
which using the functions found in \verb^neillsimplescreen.c^

To build and execute this code, use the \verb^Makefile^ provided by typing:
\begin{terminaloutput}
$ make demo_neillsimplescreen
gcc demo_neillsimplescreen.c neillsimplescreen.c
-o demo_neillsimplescreen -Wall -Wextra -pedantic 
-std=c99 -Wvla -g3 -fsanitize=undefined -fsanitize=address -lm
$ ./demo_neillsimplescreen 
\end{terminaloutput}

\begin{exercise}
Adapt any of the exercises from Chapter~4, so that output is displayed
using the simple functions demonstrated above, rather than plain text.
\end{exercise}
