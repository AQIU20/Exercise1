\nsection{ANSI Escape Sequences}
C has no inherent functionality to allow printing in colour etc.
Therefore, some terminals allow certain control codes to be printed
to move the cursor, change the foreground and background colour, and
so on.
Note that only some terminals support these control code - if you're
using a different one, garbage may appear on the screen instead.
\wwwurl{https://en.wikipedia.org/wiki/ANSI_escape_code}

I've created a very simple set of functions which shows a small
fraction of this functionality called \verb^demo_neillsimplescreen.c^
which using the functions found in \verb^neillsimplescreen.c^

To build and execute this code, use the \verb^Makefile^ provided by typing:
\begin{terminaloutput}
$ make demo_neillsimplescreen
gcc demo_neillsimplescreen.c neillsimplescreen.c
-o demo_neillsimplescreen -Wall -Wextra -pedantic 
-std=c99 -Wvla -g3 -fsanitize=undefined -fsanitize=address -lm
$ ./demo_neillsimplescreen 
\end{terminaloutput}

\begin{exercise}
Adapt the Forest Fire code in Exercise~\ref{ex:forestf} so that the
output is displayed using the simple functions demonstrated above,
with trees being green, fire being red and background being empty.
The main loop will update the forest, clear the screen, display it,
wait for a very short time (e.g. $0.10s$) and repeats.
\end{exercise}

\begin{exercise}
Adapt the wireworld code in Exercise~\ref{ex:wirew} so that the output
is displayed using this library, with tails being red, heads being blue,
copper being yellow and background being black.  The main loop will
update the board, display it, and repeat until a quit event occurs
(e.g. a mouse click or the ESC key is pressed).
\end{exercise}


\begin{exercise}
Adapt the life code in Exercise~\ref{ex:life106} so that the output is
displayed using this library, with sensible choices made for cell colours.
The main loop will update the board, display it, and repeat until a quit
event occurs (e.g. a mouse click or the ESC key is pressed).
\end{exercise}

