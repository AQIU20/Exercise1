\nsection{Submitting}

This year, you have the option of submitting one of your Chapter~2
exercises to be peer reviewed.  This process is optional - there is
no mark associated with this. However, we think it's good practice to
learn how to submit your code through Blackboard, gain an insight into
the English Universities marking system, and to think more deeply about
how the marking criteria are applied.

To submit your work, go to the Unit Blackboard page, then \emph{Assessment
Submission and Feedback}, then \emph{Peer Assessment}. You can submit
as many times as you like - old files will be automatically archived.

Make sure your file has the correct name - this will be shown on the
\emph{Unit Information} / \emph{Weekly Schedule} part of the Blackboard
page.

\nsection{Marking Criteria}

When files are ready to be marked, we'll put them in the {\emph{Files}}
folder of the \emph{General} channel of the unit `Teams' group. We'll
tell you which usernames you are assessing.

When you give feedback, you should use the following guidelines~:

\begin{itemize}
\item If the file has been submitted late then I'll rename it to have {\emph{\_late}} in the name. In this case, take 10 off the final mark (a University rule).
\item Our system will automagically prefix your username to the filename. Apart from this, if the file was misnamed in any way, even just using a \verb^.C^ rather than a \verb^.c^ extension, take 5 off the final mark.
\end{itemize}

\noindent Below, where a category has a fixed mark (e.g. choose 0 or 10), decide which of these two marks to award.
Where a category has a range of marks (e.g. 0 – 20), apply the following guidelines:
\begin{description}
\item[50\%] Just OK - room for improvement.
\item[60\%] Good, solid, work - what you might expect.
\item[70\%] Above and beyond what we would expect at this stage iof the unit (just!)
\item[80\%] We could publish this, ground-breaking - it'll change how we teach this unit in the future~!
\end{description}

\noindent So, if a category can be marked in the range 0 – 20, a very good piece of work might score 13 marks, for instance.

Using the House Style rules, shown in Appendix~\ref{appendix:style}, give a mark and comment on each of the following items:
\begin{description}
\item[FLAGS] - Does the code compile using clang and the house-style flags :\\ 
-Wall  -Wextra  -Wfloat-equal -Wvla  -pedantic  -std=c99  -O2\\
with zero warnings?
\hfill{\bf{0 or 10}}

\item[GOTO + TABS + INDENT]
\hfill{\bf{0 – 10}}

\item[BRACE]
\hfill{\bf{0 – 5}}

\item[NAMES]
\hfill{\bf{0 – 5}}

\item[LLEN]
\hfill{\bf{0 – 5}}

\item[MAIN]
\hfill{\bf{0 – 5}}

\item[CAPS]
\hfill{\bf{0 – 5}}

\item[FLEN]
Including main()
\hfill{\bf{0 – 5}}

\item[MAGIC]
\hfill{\bf{0 – 5}}

\item[BRIEF]
\hfill{\bf{0 – 5}}

\item[TYPE]
\hfill{\bf{0 – 5}}

\item[$*$] Are all functions thoroughly tested using a function called test() or similar?
\hfill{\bf{0 – 20}}

\item[$*$] Does the program work correctly?
\hfill{\bf{0 or  5}}

\end{description}

\noindent Write some brief feedback about the program(s) you've been
assigned and email it to those people using their usernames.  An example
is given below:

\begin{quote}
This program compiled for me without warnings and was
correctly named.  The code was consistently indented, but you used tabs
not spaces as required.  Function names could have been improved upon -
e.g. \verb^doit()^ is not very helpful. One function (\verb^extra()^)
seems quite long. You used a `magic number' : 65. The testing was quite
thorough and the program worked well.  Overal I'd give this 59\% - nicely
done, but pay closer attention to the style guidelines!
\end{quote}
