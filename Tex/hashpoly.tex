\nsection{Polymorphic Hashing}

Polymorphism is the concept of writing functions (or ADTs), without
needing to specify which particular type is being used/stored. To
understand the quicksort algorithm, for instance, doesn't really require
you to know whether you're using integers, doubles or some other type. C
is not very good at dealing with polymorphism - you'd need something
like Java or C++ for that. However, it does allow the use of
\verb^void*^ pointers for us to approximate it.

\begin{exercise}
\label{ex:hahspoly}
Here we write an implementation of a polymorphic associative array using
hashing, see \verb^assoc.h^. Both the key \& data to be used by the hash
function are of
unknown types, so we will simply store void pointers to
both of these, and the user of the associative array (and {\bf not} the
ADT itself) will be responsible for creating and maintaining such memory,
and also ensuring it doesn't change when in use by the associative array.
In the \verb^testassoc.c^ file we show two uses for such a type :
a simple string example (to find the longest word in English that is
also a valid (but different) word when spelled backwards), and a simple
integer example where we keep a record of how many unique random numbers
in a range are chosen.

Since we do not know {\bf what} the type of the key is, we
need to be careful when comparing or copying keys. Therefore, in the
\verb^assoc_init(int keysize)^ function, the user has to pass the size of the
key used (e.g. \verb^sizeof(int)^) or in the case or strings, the special value $0$.
Now we can use \verb^memcmp()^ and \verb^memcpy()^, or in the case of strings, \verb^strcmp()^
and \verb^strcpy()^ for dealing with the keys. In the case of data, the ADT only ever needs
to return a pointer to the data (not process it) via
\verb^assoc_lookup()^, so its size is not important.

Your hash table used to implement the ADT should be resizeable, and you may use either
open-addressing or double-hashing to deal with collisions. Make no assumptions about
the maximum size of the array, and make the initial size of the array $16$.
You can use any hash function you wish, but if it's off the internet (etc.) cite
the source in a comment..

Submit a single \verb^.zip^ file containing your code.
Your standard submission will contain two files:\\
\verb^Realloc/specific.h^ and \verb^Realloc/realloc.c^.
I will use my \verb^assoc.mk^ Makefile to compile your code,
so check that it works correctly.
\end{exercise}

Hint : when you do a resize you cannot simply
copy the old table, but must rehash your data, one entry at a time,
into the new table.  This is because your hash function is based on the
table size, and if the table size has changed you will be hashing keys
into different locations.

\nsection{Cuckoo Hashing}

\begin{exercise}
Rewrite your associative array described in Exercise~\ref{ex:hahspoly}
using cuckoo hashing instead.
This involves two tables, and a key is guaranteed to be found (if it
has been inserted) in one of its two locations. Different hash functions
must be used for each of the two tables.
If you do the extension, submit:\\
\verb^Cuckoo/specific.h^ and \verb^Cuckoo/cukcoo.c^.
\end{exercise}
