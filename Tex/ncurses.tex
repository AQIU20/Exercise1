\nsection{ncurses}
C has no inherent functionality to allow printing in colour etc.
Therefore, a programming library know a \verb^ncurses^ was created in 1993
to allow terminals to interpret certain control-codes as colours and other effects. 

The library itself is somewhat complex, allowing keyboard and mouse events to
be captured and a whole range of simple graphics functionality.
On the web page is my `wrapper' for the library, along with a program demonstrating its use.
This will only work in unix-style terminals. 
Note that after you begin ncurses mode (using \verb^Neill_NCURS_Init()^) that
you can't print to stdout or stderr, until you switch it off ~(using \verb^Neill_NCURS_Done()^).

To compile the code you'll have to use both my code \verb^neillncurses.c^
and also link in the ncurses library. A typical compile might look like
\begin{terminaloutput}
gcc yourcode.c neillncurses.c -Wall -Wfloat-equal -Wextra -O2
    -Wvla -pedantic -std=c99 -lncurses -lm
\end{terminaloutput}

If you're running a virtual box you may also need to install the \verb^ncurses^ developer
files, including \verb^ncurses.h^, using:
\begin{terminaloutput}
sudo apt install libncurses-dev
\end{terminaloutput}

Some terminals do not support ncurses, so make sure you are using an `xterm' or equaivalent.
