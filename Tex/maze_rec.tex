\nsection{Maze}

\newexercise{2023}

\newcommand{\W}{|[fill=ocre,text=white]|\#}
\newcommand{\G}{|[fill=green,text=white]|+}
\newcommand{\R}{|[fill=white,text=gray]|.}

\begin{tikzpicture}[every node/.style={anchor=base,text depth=.5ex,text height=2ex,text width=1em,outer sep=0pt,align=center,inner sep=0pt}]
\matrix [matrix of nodes,draw=white,nodes in empty cells]
{
\W&\W&\W&\W&\W&\W&\W&\W&\W&\W\\
\R&\R&\W&\R&\R&\R&\R&\R&\R&\W\\
\W&\R&\W&\R&\W&\R&\W&\W&\R&\W\\
\W&\R&\W&\R&\W&\W&\W&\W&\R&\W\\
\W&\R&\W&\R&\R&\R&\R&\W&\R&\W\\
\W&\R&\W&\R&\W&\W&\W&\W&\R&\W\\
\W&\R&\W&\R&\R&\R&\R&\W&\R&\W\\
\W&\R&\W&\W&\W&\W&\R&\W&\R&\W\\
\W&\R&\R&\R&\R&\R&\R&\W&\R&\R\\
\W&\W&\W&\W&\W&\W&\W&\W&\W&\W\\
};
\end{tikzpicture}

Escaping from a maze can be done in several ways (ink-blotting, righthand-on-wall etc.)
but here we look at recursion.

\begin{exercise}
\label{ex:maze_rec}
Write a program to read in a maze typed by a user via the filename passed to \verb^argv[1]^.
You can assume the maze will be no larger than $20 \times 20$,
walls are designated by a \verb^#^ and the rest are
spaces. The entrance can be assumed to be the gap in the wall closest to
(but not necessarilty exactly at) the top lefthand corner.
The sizes of the maze are given on the first line of the file (width,height).
Write a program that finds the route through a maze, read from this file,
and prints out the solution (if one exists) using full stops.
If the program succeeds it should exit with a status of \verb^0^,
or if no route exists it should exit with a status of \verb^1^.
\end{exercise}


\noindent It becomes obvious that the walls of every maze (having one unique solution)
must consist of two separate sections~:

\renewcommand{\W}{|[fill=ocre,text=white]|B}
\renewcommand{\G}{|[fill=orange,text=white]|A}
\renewcommand{\R}{|[fill=white,text=gray]|.}
\begin{tikzpicture}[every node/.style={anchor=base,text depth=.5ex,text height=2ex,text width=1em,outer sep=0pt,align=center,inner sep=0pt}]
\matrix [matrix of nodes,draw=white,nodes in empty cells]
{
\G&\G&\G&\G&\G&\G&\G&\G&\G&\G\\
\R&\R&\G&\R&\R&\R&\R&\R&\R&\G\\
\W&\R&\G&\R&\W&\R&\W&\W&\R&\G\\
\W&\R&\G&\R&\W&\W&\W&\W&\R&\G\\
\W&\R&\G&\R&\R&\R&\R&\W&\R&\G\\
\W&\R&\G&\R&\W&\W&\W&\W&\R&\G\\
\W&\R&\G&\R&\R&\R&\R&\W&\R&\G\\
\W&\R&\G&\G&\G&\G&\R&\W&\R&\G\\
\W&\R&\R&\R&\R&\R&\R&\W&\R&\R\\
\W&\W&\W&\W&\W&\W&\W&\W&\W&\W\\
};
\end{tikzpicture}

\begin{exercise}
\label{ex:maze_two}
Write a program to read in a maze in the same manner as in Exercise~\ref{ex:maze_rec}, and
then display the two sections using the characters \verb^A^ and \verb^B^.  
\end{exercise}

