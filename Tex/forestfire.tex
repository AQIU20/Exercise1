\nsection{Forest Fire}

You can create a very simple model of forest fires using 
a cellular automaton in the form a $2D$ grid of cells.
Each cell can be in one of three states; either `empty', `tree', or `fire'.
The next generation of cells follows these rules:
\begin{itemize}
\item A `fire' cell will turn into an `empty' cell.
\item A `tree' that is within the 8-neighbourhood of a `fire' cell will itself become `fire'.
\item A `tree' will burn (due to a lightning strike) $1$ time in $L$.
\item An `empty' space will become a `tree' (spontaneous growth) $1$ time in $G$.
\end{itemize}
\noindent See also:\\
\wwwurl{https://en.wikipedia.org/wiki/Forest-fire_model}
\wwwurl{https://www.aryan.app/randomstuff/forestfire.html}

\noindent You can experiment with different values of $L$ and $G$, but
a useful starting point is $G=250$ and $L=10 \times G$.

\begin{exercise}
\label{ex:forestf}
Write a program which creates an empty $2D$ grid of cells, $80$ wide and
$30$ high.
Then, apply the rules above to iterate the simulation, so that the next generation
is created from the previous one.

Print out every generation onto the screen, using a space to represent an empty cell,
the \verb^@^ character for a tree, and \verb^*^ for fire cells.
Display the board for $1000$ generations using plain text.
You may assume that the grid is always $80$ cells by $30$

\end{exercise}
