\nsection{Packed Boolean Arrays}

In C99, every individual variable must have a different address. For
this reason, a \verb^bool^ must take up a least one byte since this is
the smallest individual element uniquely addressable by most hardware.

In some cases it would be good to have a datatype that truly uses one bit
of storage for each Boolean. One such example is known as the  {\it packed}
Boolean array, where an array of $64$ elements takes $8$ bytes.

Here we create an ADT for packed Boolean arrays, allowing logical
operations to be performed on an entire array, and individual bits to
be set and unset.

\noindent
This will involve manipulation of the bits of the data
using the C bitwise logical operators `xor' (\verb#^#), `or' (\verb^|^) and `and' (\verb^&^).
\wwwurl{https://en.wikipedia.org/wiki/Bitwise\_operation}

\begin{exercise}
Given the files \verb^testboolarr.c^ and \verb^boolarr.h^, complete the Boolean Array ADT using
a reallocating array of unsigned chars. You will need to write both \verb^realloc.c^, which will define the 
functions in boolarr.h, and \verb^specific.h^, which will contain your header information. 
Ignoring the overhead of the main structure, 
which might store the capacity of the underlying array, and the number of valid bits in use, 
the array should be around $~$\verb^nbits^$/8$ bytes in length.
\end{exercise}
