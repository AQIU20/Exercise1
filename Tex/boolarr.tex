\nsection{Boolean Arrays}

In C90 there is no real way of having Booleans.
You can \verb^typedef^ and/or \verb^enum^erate one, but in reality this is
just an integer, not truly a single bit.
Therefore, we'd like to be able to create an ADT that allows Boolean Arrays
to be created, such that the storage of a single bit, requires only one bit of memory.

\noindent
This will involve manipulation of the bits of a (for instance) \verb^unsigned char^
using the C bitwise logical operators `xor' (\verb#^#), `or' (\verb^|^) and `and' (\verb^&^).
\wwwurl{https://en.wikipedia.org/wiki/Bitwise\_operation}

\begin{exercise}
Given the files \verb^testboolarr.c^ and \verb^boolarr.h^, complete the Boolean Array ADT using
a reallocating array of unsigned chars. You will need to write both \verb^realloc.c^, which will define the 
functions in boolarr.h, and \verb^specific.h^, which will contain your header information. 
Each cell of the underlying array stores $8$ bits of the data. Ignoring the overhead of the main structure, 
which might store the capacity of the underlying array, and the number of valid bits in use, 
the array should be $~$\verb^nbits^$/8$ bytes in length.
\end{exercise}
