\nsection{Palindromes}

From \verb^wikipedia.org^~:

\begin{quote}
A palindrome is a word, phrase, number or other sequence of units that has the property of reading the same in either direction (the adjustment of punctuation and spaces between words is generally permitted).

The most familiar palindromes, in English at least, are character-by-character: the written characters read the same backwards as forwards. Palindromes may consist of a single word (such as "civic" or "level" ), a phrase or sentence ("Neil, a trap! Sid is part alien!", "Was it a rat I saw?") or a longer passage of text ("Sit on a potato pan, Otis."), even a fragmented sentence ("A man, a plan, a canal: Panama!", "No Roman a moron"). Spaces, punctuation and case are usually ignored, even in terms of abbreviation ("Mr. Owl ate my metal worm").
\end{quote}

\begin{exercise}
Write a program that prompts a user for a phrase and tells them whether 
it is a palindrome or not.  {\bf Do not} use any of the built-in
string-handling functions (\verb^string.h^), such as
\verb^strlen()^ and \verb^strcmp()^.
However, you {\bf may} use the character functions (\verb^ctype.h^), such as
\verb^islower()^ and \verb^isalpha()^.

Check you program with the following palindromes~:
\\
\verb^"kayak"^\\
\verb^"A man, a plan, a canal: Panama!"^\\
\verb^"Madam, in Eden I'm Adam,"^\\
\verb^"Level, madam, level!"^
\end{exercise}
