\nsection{Fibonacci Words using Substitution Rules}
\label{sec:fibword_subs}

\newexercise{2022}

This follows on from Exercise~\ref{ex:fibword_phi}, where we saw a
good approximation of the Fibonacci Word using~$\phi$.

To be certain of the sequence at very large values of $n$, you'd need to
construct the sequence iteratively. One interesting way of doing this
is to use a Substitution Rule. Start with a counter indexing a
single digit zero at the beginning of a \verb^bool^ array. Now at each step,
if the digit is $0$, append a $1$ and a $0$ onto the end of the array.
If the digit is a $1$ append a $0$ onto the end of the array.
Either way, move along to the next digit and repeat.
\begin{verbatim}
Array initially :
0 (index zero contains a zero therefore append 1 0)
010 (index 1 contains a one therefore append 0)
0100 (index 2 contains a zero therefore append 1 0)
010010 (index 3 ...)
\end{verbatim}


\begin{exercise}
\label{ex:fibword_subs}
Write the function:
\begin{codesnippet}
bool fibword_subs(int n)
\end{codesnippet}
which returns to $n^{th}$-digit of the sequence (which begins at $n=1$)
using an array of \verb^bool^s, and repeatedly applying the
substitution rules described above, until you have a sequence
large enough to extract the $n^{th}$-digit. 
\end{exercise}

\begin{exercise}
Write a program which experiments with the approximation of $\phi$
from Exercise~\ref{ex:fibword_phi}, to see at what value of $n$ this begins to
return incorrect values. These `correct' values can be obtained from the
function written in Exercise~\ref{ex:figword2}.

In my implementation, when I
\begin{codesnippet}
#define PHI 1.61
\end{codesnippet}
the values are wrong from the $12^{th}$ digit.
For:
\begin{codesnippet}
#define PHI 1.61803
\end{codesnippet}
the values are wrong from the $609^{th}$ digit.

\end{exercise}
