\nsection{Irrational numbers}

Neill's favourite number (don't ask why!) is $e$ which has the value
$2.71828182845904523536\ldots$ This is an example of an irrational
number - one that can only ever be {\it approximated} by the ratio
of two integers - a poor approximation of $e$ is $87$ divided by $32$,
that is $\frac{87}{32}$.

\begin{exercise}
Write a program that loops through all possible denominators (that is
the integer on the bottom, $b$ in the fraction $\frac{a}{b}$) and finds
which $a$ and $b$ pair give the best approximation to $e$.
The output of your program might look something like~:

\begin{verbatim}
271801/99990 = 2.71828182818281849364
\end{verbatim}

\noindent You need only investigate denominators $< 100,000$.
\verb^#^define the number being searched for.
Check your code works
for other famous constants such as $\pi = 3.14159265358979323846\ldots$~:

\begin{verbatim}
312689/99532 = 3.14159265361893647039
\end{verbatim}
\end{exercise}
