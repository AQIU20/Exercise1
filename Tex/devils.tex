\nsection{The Devil's Dartboard}

\tikzstyle{wired}=[draw=gray!30, line width=0.15mm]
\tikzstyle{number}=[anchor=center, color=white]
% Sectors are numbered 0-19 counterclockwise from the top.

% \strip{color}{sector}{outer_radius}{inner_radius}
\newcommand{\strip}[4]{
    \filldraw[#1, wired]
      ({18 *  #2}      :                   #3) arc
      ({18 *  #2}      : {18 * (#2 + 1)} : #3) --
      ({18 * (#2 + 1)} :                   #4) arc
      ({18 * (#2 + 1)} : {18 *  #2}      : #4) -- cycle;
}

% \sector{color}{sector}{radius}
\newcommand{\sector}[3]{
    \filldraw[#1, wired]
      (0, 0) --
      ({18 * #2} :                   #3) arc
      ({18 * #2} : {18 * (#2 + 1)} : #3) -- cycle;
}

\newcommand{\dartboard}{
 % These are the official dartboard dimensions as per BDO's regulations.
  % The whole board's background.
  \fill[black] (0, 0) circle (225.5mm);
  % Even sections.
  \foreach\i in {0,2,...,18} {
    \sector{black}{\i}{162mm}
    \strip{red}{\i}{170mm}{162mm} % Double strip.
    \strip{red}{\i}{107mm}{ 99mm} % Treble strip.
  }
  % Odd sections.
  \foreach\i in {1,3,...,19} {
    \sector{white}{\i}{162mm}
    \strip{green}{\i}{170mm}{162mm} % Double strip.
    \strip{green}{\i}{107mm}{ 99mm} % Treble strip.
  }
  % Bull's ring and eye.
  \filldraw[green, wired] (0, 0) circle (15.9mm);
  \filldraw[red,   wired] (0, 0) circle (6.35mm);
 % Labels.
  \foreach \sector/\label in {%
      0/20,  1/ 1,  2/18,  3/ 4,  4/13,
      5/ 6,  6/10,  7/15,  8/ 2,  9/17,
     10/ 3, 11/19, 12/ 7, 13/16, 14/ 8,
     15/11, 16/14, 17/ 9, 18/12, 19/ 5}%
  {
    \node[number] at ({18 * (-\sector + .5)} : 197.75mm) {\sffamily\label};
  }
}


\begin{figure}[h]
\centering
\begin{tikzpicture}[scale=.10]
\begin{scope}[rotate=81]
\dartboard
\end{scope}
\begin{scope}
\node[text=black] (trb) at (-200mm, 300mm) {Treble 20};
\draw [-Circle,ocre] (trb.east) -- (-0mm,100mm);
\node[text=black] (dbl) at (175mm, 300mm) {Double 4};
\draw [-Circle,ocre] (dbl.south) -- (135mm,95mm);
\end{scope}
\end{tikzpicture}
\end{figure}

In the traditional `pub' game, darts, there are 62 different possible scores~:
single 1 - 20 (the white and black areas), double 1 - 20 (the outer red and green segments)
(i.e. 2, 4, 6, 8 $\ldots$), treble 1 - 20 (i.e. 3, 6, 9, 12 $\ldots$) (the inner red or green segments), 25 (small green circle) and 50 (the small red inner circle).


It's not obvious, if you were inventing darts from scratch, how best to
lay out the numbers. The London board shown seems to have small numbers near high numbers,
so that if you just miss the $20$ for example, you'll hit a small number instead.

Here we look at a measure for the `difficulty' of a dartboard.
One approach is to simply sum up the values of adjacent triples, squaring 
this number. So for the London board shown, this would be:
\begin{math}
(20+1+18)^2 + (1+18+4)^2 + (18+4+13)^2 \dots (5+20+1)^2 = 20478
\end{math}

For our purposes a {\bf lower} number is better\footnote{It's beyond
the scope here to explain why!}. For more details see~:
\wwwurl{http://www.mathpages.com/home/kmath025.htm}

\begin{exercise}
Write a program that repeatedly chooses two
positions on the board at random and swaps them. If this leads to a lower cost,
keep the board. If not, unswap them. Repeat this {\it greedy search}
$5000000$ times, and print out the best board found.
Begin with the trivial monotonic sequence.
The output may look something like~:
\begin{terminaloutput}
Total = 19966 :  3 19 11  2 18 12  1 20 10  4 16  8
14  5 13 15  6  7 17  9 
\end{terminaloutput}
or
\begin{terminaloutput}
Total = 19910 :  3 18 10  5 16  9  8 14 11  4 19  6
7 20  2 13 15  1 17 12 
\end{terminaloutput}

Is the score of $19874$ the lowest possible that may be obtained via this technique~?
\end{exercise}

%\begin{exercise}
%You can also argue that the $16$ next to the $8$ is a bad idea. Alternate
%odd-and-even numbers is appealing.
%Is~:
%\begin{terminaloutput}
%Total = 19886 : 20  3 10 19  2 11 18  1 14 13  4 15 12  5 16  9  6 17  8  7 
%\end{terminaloutput}
%the best possible ?
%\end{exercise}
%
%
%\wwwurl{https://cms.math.ca/crux/v26/n4/page215-217.pdf}
%
%Here is a polynomial-time algorithm that, if it does not always produce a
%Devil's Dartboard, seems to come pretty close, especially for large $n$.
%The permutation p is defined by induction. Choose p(k + 1) such that the
%3-sum $p(k-1) + p(k) + p(k + 1)$ is as close as possible to the mean $m =
%3(n + 1)/2$ of the 3-sums. In the case of a tie, select $p(k + 1)$ so that
%consecutive 3-sums (different from $m$) fall on opposite sides of $m$,
%beginning with a 3-sum greater than $m$. To get the induction started,
%set $p(1) = n$ and $p(2) = 1$.
%
%For example, if $n = 6$, we have $m = 3(6+ 1)/2 = 10.5$. The algorithm
%then produces the Devil's Dartboard 6, 1, 4, 5, 2, 3.
%
%For $n = 20$, the algorithm generates the permutation $20, 1, 11, 19,
%2, 10, 18, 4, 9, 17, 6, 8, 16, 7, 12, 13, 5, 14, 15, 3$ which is not as
%good as the traditional London board.




