\nsection{Binary Grid Puzzle}
Many newspapers in the UK have a puzzles section.
One of that has become popular recently is the {\em Binary Grid}
\wwwurl{https://www.binarypuzzle.com/}

A Binary Grid is a $2D$ square grid of cells, each of which is either empty, or contains the value $0$ or~$1$.
A set of rules governs what is legal in a {\bf valid} board~:
\begin{itemize}
\item Each cell should contain either a zero or a one.
\item No more than two of the same number can appear consecutively in a row or a column.
\item There must be an equal number of $0s$ and $1s$ in each row and column - this means that the height/width of the
puzzle must be even.
\end{itemize}

\noindent Given these rules\footnote{In
the full version of this puzzle, no row or column can be the same - we ignore this constraint here.}
we can deduce some principles to help us find a solution~:\\
\begin{description}
\item[Pairs] : If two of the same number ($0s$ or $1s$) are in adjacent cells (either horizontally or vertically) then the two surrounding
cells on either side must be the other number, otherwise there would be three of the same number in sequence.
\item[OXO] : If there are two of the same number with a gap of one empty cell between them (horizontally or vertically),
then the opposite number must be in the empty cell (otherwise we'd have three of the same number in sequence).
\item[Counting] : If all of a particular number have been found in a row (or column) - that is $3$ if the grid is a $6x6$, then the remaining $3$ unknown cells must all contain the other number.
\end{description}

\noindent An example puzzle might look so~:\\
\begin{center}
\begin{tikzpicture}
\matrix[matrix of nodes,nodes={draw=black, anchor=center, minimum size=.6cm,fill=ocre!30}, column sep=-\pgflinewidth, row sep=-\pgflinewidth, , execute at empty cell={\node[draw=black,text=black,fill=gray!20]{.};} ] (A) {
 & & & & &0\\
0& & &1&1& \\
 & & &1& & \\
 & & & & &0\\
 & & & & & \\
 & & & & & \\
};
\end{tikzpicture}
\end{center}

\noindent By applying the {\bf Pairs} principle $4$ times we get~:\\
\begin{center}
\begin{tikzpicture}
\matrix[matrix of nodes,nodes={draw=black, anchor=center, minimum size=.6cm,fill=ocre!30}, column sep=-\pgflinewidth, row sep=-\pgflinewidth, , execute at empty cell={\node[draw=black,text=black,fill=gray!20]{.};} ] (A) {
 & & &0& &0\\
0& &0&1&1&0\\
 & & &1& & \\
 & & &0& &0\\
 & & & & & \\
 & & & & & \\
};
\end{tikzpicture}
\end{center}

\noindent The {\bf OXO} principle gives~:\\
\begin{center}
\begin{tikzpicture}
\matrix[matrix of nodes,nodes={draw=black, anchor=center, minimum size=.6cm,fill=ocre!30}, column sep=-\pgflinewidth, row sep=-\pgflinewidth, , execute at empty cell={\node[draw=black,text=black,fill=gray!20]{.};} ] (A) {
 & & &0& &0\\
0&1&0&1&1&0\\
 & & &1& & \\
 & & &0& &0\\
 & & & & & \\
 & & & & & \\
};
\end{tikzpicture}
\end{center}

\noindent Application of the Counting principle can now fill the rightmost column~:\\
\begin{center}
\begin{tikzpicture}
\matrix[matrix of nodes,nodes={draw=black, anchor=center, minimum size=.6cm,fill=ocre!30}, column sep=-\pgflinewidth, row sep=-\pgflinewidth, , execute at empty cell={\node[draw=black,text=black,fill=gray!20]{.};} ] (A) {
 & & &0& &0\\
0&1&0&1&1&0\\
 & & &1& &1\\
 & & &0& &0\\
 & & & & &1\\
 & & & & &1\\
};
\end{tikzpicture}
\end{center}

\noindent Some (but not all) puzzles can be solved by repeatedly applying these three rules.
This is what we explore in this assignment.

\begin{exercise}

Code for this exercise can be found in \wwwurl{https://github.com/csnwc/Exercises-In-C}
\noindent then navigate into \verb^Code/Week4/BinaryGrid^.

Complete the file {\bf bingrid.c} which, along with my files {\em
bingrid.h} and {\em bingrid\_driver.c}, allows Binary Grids to be solved.

\noindent My file {\em bingrid.h} contains the function definitions
which you'll have to implement in your {\bf bingrid.c} file.  My file
{\em bingrid\_driver.c} contains the \verb^main()^ function to act as
a driver to run the code.  Your file will contain many other functions
as well as those specified, so you'll wish to test them as normal using
the \verb^test()^ function.

\noindent If all of these files are in the same directory, you can
compile them using the \verb^Makefile^ given.

\noindent Do not alter or resubmit {\em bingrid.h} or {\em
bingrid\_driver.c} - my original versions will be used to compile the
{\em bingrid.c} file that you adapt.

\end{exercise}
