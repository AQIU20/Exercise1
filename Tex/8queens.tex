\nsection{The N-Queens Puzzle}

In the game chess, the piece known as the Queen can take other pieces horizontally, vertically or diagonally~:

\begin{center}
\chessboard[
  setwhite={Qd3},
  pgfstyle=straightmove,
  arrow=stealth,
  linewidth=.2ex,
  padding=1ex,
  color=ocre!50,
  pgfstyle=straightmove,
  shortenstart=1ex,
  showmover=false,
  markmoves={d3-h7,d3-a6,d3-b1,d3-f1,d3-d8,d3-d1,d3-a3,d3-h3},
]
\end{center}

The eight queens puzzle is the problem of placing eight chess queens
on an $8\times 8$ chessboard so that no two queens threaten each other; thus,
a solution requires that no two queens share the same row, column,
or diagonal.
\wwwurl{https://en.wikipedia.org/wiki/Eight_queens_puzzle}

\noindent One possible solution for an $8\times 8$ board is~:

\begin{center}
\chessboard[
  setwhite={Qa8,Qb4,Qc1,Qd3,Qe6,Qf2,Qg7,Qh5},
  showmover=false,
]
\end{center}

\noindent Since there must be one, and one only, queen in each file
(column), the above board can be summarized using the queen's position
on each rank (height) - the board above would be numbered $84136275$.

\noindent
Since finding all solutions may take a long time, it is normal to allow for board sizes other than $8\times 8$ (but always square) by specifying the size of the board, $n$.

\begin{exercise}
Write a program to find solutions to the $n$ queens problem, using a brute-force algorithm to find all possible solutions.
You will use an array (list) of structures, each one containing the data for one board~:
\begin{enumerate}
\item Put the initial (empty) board into the front of this list, \verb^f=0^.
\item Consider the board at the {\bf front} of the list (index \verb$f$).
\item For this (parent) board, find the resulting (child) boards which can be created by
placing one queen into the board into a position that doesn't check any other queens. For each of these
child boards~:
\begin{itemize}
\item If a child is unique (i.e.\ it has not been seen before in the list), add it to the end of the list.
\item If it has been seen before (a duplicate) ignore it.
\item If it is a solution board, keep a record of the number of them (and possibly) print out a summary of this board.
\end{itemize}
\item Add one to $f$. If there are more boards in the list, go to step $2$.
\end{enumerate}

\noindent
We will allow for board sizes other than $8\times 8$, but
{\bf never} greater than $10\times 10$, as specified on the command line~:
\begin{terminaloutput}
$ ./8q 6
4 solutions
\end{terminaloutput}

If the `verbose' flag is used, your program will also print out summaries of each solution~:
\begin{terminaloutput}
$ ./8q -verbose 6
362514
246135
531642
415263
4 solutions
\end{terminaloutput}
(Since $10 \times 10$ boards may be printed, we'll use the numbers $1 \ldots 9$ and also $A$).

\noindent
Your program~:
\begin{itemize}
\item {\bf Must} use the algorithm detailed above (which is similar to a queue and therefore a breadth-first search). Do not use the other algorithms possible (e.g. best-first, permutations, etc.); the quality of your coding is being assessed against others taking the same approach.
\item {\bf Must not} use dynamic arrays or linked lists. Since boards cannot be any larger than $10 \times 10$, you can create boards of this size, and only use a sub-part of them if the board required is smaller. The list of boards can be a fixed (large) size.
\item {\bf Should} check for invalid board sizes specified, and report in a graceful way if there is a problem, aborting with \verb^exit(EXIT_FAILURE)^ if so.
\item {\bf Should not} print anything else out to screen after successfully
completing the search, except that which is shown above. Automated checking
will be used during marking, and therefore the output must be precise.
\item {\bf Should} call the function \verb^test()^ to perform any assertion testing etc.
\item {\bf Should} be submitted by uploading a single \verb^8q.zip^ file to Blackboard, which will include (at least) \verb^8q.c^ and \verb^Makefile^ which allows me to compile your code by typing \verb^make 8q^ which creates \verb^8q^, the executable file and can be run using \verb^make run^.
\end{itemize}

\subsection*{Extension}

Basic assignment = {\Large $90\%$}.
Extension = {\Large $10\%$}.

\noindent
If you'd like to try an extension, make sure to put {\it extension.c} and an explanation of it in {\it extension.txt}
as part of your submission. Update your \verb^Makefile^ so that we can build your code using \verb^make extension^ and run it with \verb^make extrun^.
\noindent The extension could
involve a faster search technique, better graphical display, user input
or something else of your choosing.

\end{exercise}
