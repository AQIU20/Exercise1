\nsection{Cons, Car and Cdr}
\label{ex:carcdr}

When storing real-world data, lists containing other lists are ubiquitous e.g.~:
\verb^(0 (1 2) 3 4 5)^.

\noindent In fact many functional (or at least partly functional) languages are
based around the idea that all data are, conceptually, nested lists. Examples
of such languages include Haskell and, of interest here, Lisp~:
\wwwurl{https://en.wikipedia.org/wiki/Lisp\_(programming_language)}

One of the main operations of these languages is the ability to extract
the front (head) or remainder (tail) of the list {\bf very} efficiently.   
A traditional linked list is not good for this, since it would require work
to untangle the head (it has a pointer to the rest of the list) and
it's unclear how a list-within-a-list would be stored.

For Lisp, a better structure was developed, which is slightly more complex than
a traditional linked list, but which makes the extraction of the head (and remainder)
simple and fast. To encode the list \verb^(1 2)^, which is a list containing two atoms,
we use~:
\begin{terminaloutput}
[*|*]--->[*|*]--->NIL
 |        |
 v        v
 1        2
\end{terminaloutput}

\noindent Here the main struct (known as the {\it cons})
has two pointers.  The ones going
downwards known as the {\it car} pointers and the ones going horizontally
the {\it cdr}\footnote{ From wiki: Two assembly language macros $...$
became the primitive operations for decomposing lists: {\it car} (Contents
of the Address part of Register number) and {\it cdr} (Contents of the
Decrement part of Register number).  } pointers.
If we were to have a pointer $p$ to the first {\it cons} of this list, the head of the
list is pointed to by \verb^p->car^, and the remainder of the list simply by \verb^p->cdr^.

\newpage 

A more complex list, \verb^(0 (1 2) 3 4 5)^ would be stored as~:
\begin{terminaloutput}
[*|*]--->[*|*]------------------>[*|*]--->[*|*]--->[*|*]--->NIL
 |        |                       |        |        |
 v        v                       v        v        v
 0       [*|*]--->[*|*]--->NIL    3        4        5
          |        |
          v        v
          1        2
\end{terminaloutput}

\noindent The atomic elements (the integers) are stored as `leaf' nodes with both {\it car} and {\it cdr}
pointers set to NULL. The data structure used to store the atom and also {\it car} and {\it cdr} pointers
is known as a {\it cons} (short for constructor), due the to process by which we build lists.

So, here, we're interested in recreating this data structure, allowing the user to build lists (the
{\it cons} operation), extract the head and remainder of lists (the {\it car} and {\it cdr} operations)
and other associated functions such as copying a list or counting the number of elements in it.

\begin{exercise}
Use a dynamic/linkedlist style approach to implement a car/cdr ADT. Write the source files
\verb^Linked/specific.h^ and \verb^Linked/linked.c^, so that they compile against my
\verb^lisp.h^ and \verb^testlisp.c^ files and run successfully using my \verb^Makefile^.
The basic operations are~:
\begin{verbatim}
lisp* lisp_atom(const atomtype a);
lisp* lisp_cons(const lisp* l1,  const lisp* l2);
lisp* lisp_car(const lisp* l);
lisp* lisp_cdr(const lisp* l);
atomtype lisp_getval(const lisp* l);
bool lisp_isatomic(const lisp* l);
lisp* lisp_copy(const lisp* l);
int lisp_length(const lisp* l);
void lisp_tostring(const lisp* l, char* str);
void lisp_free(lisp** l);
\end{verbatim}

\noindent This is worth $90\%$ of the marks.
Additional functions you can implement, worth $10\%$, are~:
\begin{verbatim}
lisp* lisp_fromstring(const char* str);
lisp* lisp_list(const int n, ...);
void lisp_reduce(void (*func)(lisp* l, atomtype* n), lisp* l, atomtype* acc);
\end{verbatim}
and are {\bf Exensions} worth $10\%$ of the marks.

\noindent Even if you don't get these extensions (or other functions) to work correctly, make
sure the code still compiles by writing `dummy' functions as placeholders,
even if some of the assertions fail.
\end{exercise}
