\nsection{Music Playlisters}

Most MP3 players have a ``random'' or ``shuffle'' feature.
The problem with these is that they can sometimes be {\bf too} random;
a particular song could be played twice in succession if the new song to
play is truly chosen randomly each time without taking into account what has already been played.

To solve this, many of them randomly order the entire playlist so that
each song appears in a random place, but once only.
The output might look something this:

\begin{terminaloutput}
How many songs required ? 5
4 3 5 1 2
\end{terminaloutput}

or~:

\begin{terminaloutput}
How many songs required ? 10
1 9 10 2 4 7 3 6 5 8
\end{terminaloutput}

\begin{exercise}\label{ipod:ex_a}
Write a program that gets a number from the user (to represent the number
of songs required) and outputs a randomised list.
\end{exercise}

\begin{exercise}
Rewrite Exercise~\ref{ipod:ex_a} so that the program passes an array of integers (e.g. \verb^[1,2,3,4,5,6,7,8,9,10]^) to a function which shuffles them {\bf in-place} (no other arrays are used) and with an algorithm having complexity $O(n)$. 
\end{exercise}
