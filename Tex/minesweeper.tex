\nsection{Minesweeper}

The game {\it Minesweeper}
\wwwurl{https://en.wikipedia.org/wiki/Minesweeper_(video_game)}
is a logic puzzle game, played on a two-dimensional
grid of squares. There are `mines' hidden in the grid, and other, numbered squares,
tell you how many mines there are in that square's (eight-count) Moore neighbourhood.
\wwwurl{https://en.wikipedia.org/wiki/Moore_neighborhood}

In our version of the game, we'll be
using just two rules to `solve' the grid by working out the unknown squares.
\\[1em]
{\bf Rule $1$}: If we've discovered all the mines on the board already, then any unknown cell
can simply be numbered with the count of mines in its Moore neighbourhood.

In the follwing grid,
if we know the total number of mines in the grid is five and these have all been found:\\
\begin{center}
\begin{tikzpicture}
\matrix[matrix of nodes,nodes={draw=black, anchor=center, minimum size=.6cm,fill=gray!10}, column sep=-\pgflinewidth, row sep=-\pgflinewidth, , execute at empty cell={\node[draw=black,text=black,fill=ocre!40]{?};} ] (A) {
0&1&1& &0\\
1&3&\textcolor{red}{X}&3&1\\
1&\textcolor{red}{X}&\textcolor{red}{X}&\textcolor{red}{X}&1\\
1&3&\textcolor{red}{X}&3&1\\
0&1&1&1&0\\
};
\end{tikzpicture}
\end{center}

\noindent then the solution to the unknown square must be:\\
\begin{center}
\begin{tikzpicture}
\matrix[matrix of nodes,nodes={draw=black, anchor=center, minimum size=.6cm,fill=gray!10}, column sep=-\pgflinewidth, row sep=-\pgflinewidth, , execute at empty cell={\node[draw=black,text=black,fill=ocre!40]{.};} ] (A) {
0&1&1&1&0\\
1&3&\textcolor{red}{X}&3&1\\
1&\textcolor{red}{X}&\textcolor{red}{X}&\textcolor{red}{X}&1\\
1&3&\textcolor{red}{X}&3&1\\
0&1&1&1&0\\
};
\end{tikzpicture}
\end{center}

{\bf Rule $2$}: For an known, numbered square $n$, with $m$ unknowns and $n-m$ mines in its Moore neighbourhood,
then those unknown squares must be mines.

\noindent Applying Rule $2$ to this grid:\\
\begin{center}
\begin{tikzpicture}
\matrix[matrix of nodes,nodes={draw=black, anchor=center, minimum size=.6cm,fill=gray!10}, column sep=-\pgflinewidth, row sep=-\pgflinewidth, , execute at empty cell={\node[draw=black,text=black,fill=ocre!40]{?};} ] (A) {
0&1&1&1&0\\
1&3&\textcolor{red}{X}&3&1\\
1&\textcolor{red}{X}&\textcolor{red}{X}&\textcolor{red}{X}&1\\
1&3& &3&1\\
0&1&1&1&0\\
};
\end{tikzpicture}
\end{center}

\noindent at the middle square on the bottom row (for instance), yields:\\
\begin{center}
\begin{tikzpicture}
\matrix[matrix of nodes,nodes={draw=black, anchor=center, minimum size=.6cm,fill=gray!10}, column sep=-\pgflinewidth, row sep=-\pgflinewidth, , execute at empty cell={\node[draw=black,text=black,fill=ocre!40]{.};} ] (A) {
0&1&1&1&0\\
1&3&\textcolor{red}{X}&3&1\\
1&\textcolor{red}{X}&\textcolor{red}{X}&\textcolor{red}{X}&1\\
1&3&\textcolor{red}{X}&3&1\\
0&1&1&1&0\\
};
\end{tikzpicture}
\end{center}

\noindent {\bf Repeated} application of these rules will allow some (but not all) boards to be solved.

\begin{exercise}

Code for this exercise can be found in \wwwurl{https://github.com/csnwc/Exercises-In-C}
\noindent then navigate into \verb^Code/Week4/Minesweeper^.

Complete the file {\em ms.c} which, along with my files {\em
ms.h} and {\em drv.c}, allows the puzzles to be solved.

\noindent My file {\em ms.h} contains the function definitions
which you'll have to implement in your {\bf ms.c} file.  My file
{\em drv.c} contains the \verb^main()^ function to act as
a driver to run the code.  Your file will contain many other functions
as well as those specified, so you'll wish to test them as normal using
the \verb^test()^ function.

\noindent If all of these files are in the same directory, you can
compile them using the \verb^Makefile^ given.

\noindent Do not alter or resubmit {\em ms.h} or {\em
drv.c} - my original versions will be used to compile the
{\em ms.c} file that you create. Only submit {\em ms.c}

\end{exercise}
