\nsection{Roman Numerals}

Adapted from:\\
\wwwurl{http://mathworld.wolfram.com/RomanNumerals.html}

\begin{quote}
``Roman numerals are a system of numerical notations used by the
Romans. They are an additive (and subtractive) system in which
letters are used to denote certain "base" numbers, and arbitrary
numbers are then denoted using combinations of symbols.
Unfortunately, little is known about the origin of the Roman numeral
system.

The following table gives the Latin letters used in Roman numerals
and the corresponding numerical values they represent~:
\end{quote}

\begin{center}
\begin{tabular}{|l|l|}\hline
I   &   1\\
V   &   5\\
X   &   10\\
L   &   50\\
C   &  100\\
D   &  500\\
M   & 1000\\
\hline
\end{tabular}
\end{center}

\begin{quote}
For example, the number 1732 would be denoted MDCCXXXII in Roman
numerals. However, Roman numerals are not a purely additive number
system. In particular, instead of using four symbols to represent a
4, 40, 9, 90, etc. (i.e., IIII, XXXX, VIIII, LXXXX, etc.), such
numbers are instead denoted by preceding the symbol for 5, 50, 10,
100, etc., with a symbol indicating subtraction. For example, 4 is
denoted IV, 9 as IX, 40 as XL, etc.''
\end{quote}

It turns out that every number between 1 and 3999 can be represented
as a Roman numeral made up of the following one- and two-letter
combinations:

\begin{center}
\begin{tabular}{|l|l|l|l|}\hline
I   &   1  & IV  &    4\\
V   &   5  & IX  &    9\\
X   &   10 & XL  &   40\\
L   &   50 & XC  &   90\\
C   &  100 & CD  &  400\\
D   &  500 & CM  &  900\\
M   & 1000 &     &     \\ \hline
\end{tabular}
\end{center}

\begin{exercise}
Write a program that contains a function which is passed a string and returns an integer.
The string is a roman numeral in the range
\verb^1 - 3999^. Amongst
others, \verb^assert()^ test that \verb^MCMXCIX^ returns \verb^1999^,
\verb^MCMLXVII^ returns \verb^1967^ and that \verb^MCDXCI^ returns
\verb^1491^.
\end{exercise}
