
\nsection{The UNIX awk program}

Sometimes handling files containing numerical data in C
may be somewhat arduous. A `simple' program to swap the first
and second columns of a file is quite long in C.

For this reason, there is a simple language called awk which
allows simple manipulation to be done on a line by line basis.
For example:
\begin{verbatim}
{
print $2, $1;
}
\end{verbatim}
swaps the first and second fields in a file.
The whole program is applied to every line of the input file in turn.

Other examples of awk would include printing fields in reverse order:
\begin{verbatim}
{ for (i = NF; i > 0; --i) print $i }
\end{verbatim}

\noindent Adding up first column, print sum and average
\begin{verbatim}
{ s += $1 }
END{print "sum is",s," average is",s/NR}
\end{verbatim}

\noindent Notice the `special' variables \verb^NR^ and \verb^NF^.
Other special variables include \verb^FS^ which defines how
fields are separated (default space and tab).
See `man awk' for more information.

\subsection*{The CAWK formal grammar}

This assignment involves writing a recursive descent parser
for a simple, cut-down version of awk called `cawk'. It is based
on the following formal grammar:

{\small
\begin{verbatim}
<PROG>       := "{" <ILST> |
                "{" <ILST> "END" "{" <ILST>
<ILST>       := <INSTR> <ILST> | "}"
<INSTR>      := <LET> | <IF> | <PRINT> | <FOR>
<LET>        := "LET" <USER> "=" <POLISH> 
<USER>       : = "$A" | .. | "$Z"
<RD>         := <USER> | "$[" <INDEX> "]" | <SPEC>
<INDEX>      := number | <USER> | <SPEC>
<SPEC>       := "$#" | "$@"
<NUMVAR>     := number | <RD>
<POLISH>     := <OPER><POLISH> | ";"
<OPER>       := <NUMVAR> | <OPERATOR>
<OPERATOR>   := "+" | "-" | "*" | "/"
<IF>         := "IF" "(" <TEST> ")" "{" <ILST>
<TEST>       := <NUMVAR> <COND> <NUMVAR>
<COND>       := "<" | ">" | "==" | "!="
<PRINT>      := "PRINT" <NUMVAR> | "PRINT" string
<FOR>        := "FOR" <USER> "=" <NUMVAR> "TO"
                <NUMVAR> "STEP" <NUMVAR> "{" <ILST>
\end{verbatim}

\noindent Note, you may assume that the program consists of
a list of words separated by white-space.
\begin{verbatim}
$# is the number of records.
$@ is the number of fields in the current record.
$[ 0 ] is the entire line
$[ i ] is the ith field of the record 
\end{verbatim}
}

\section*{Examples of CAWK}
\begin{verbatim}
{
PRINT $[ 2 ]
PRINT $[ 1 ]
PRINT "\n"
}
\end{verbatim}

\noindent Print fields in reverse order:
\begin{verbatim}
{
   FOR $I = $@ TO 1 STEP -1 {
      PRINT $[ $I ]
   }
   PRINT "\n"
}
\end{verbatim}
