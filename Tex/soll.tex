\nsection{Self-Organising Linked Lists}

A self-organising linked list (SOLL) improves search efficiency (over
an unsorted list) by rearranging the elements in the list each time they
are accessed: 

\wwwurl{https://en.wikipedia.org/wiki/Self-organizing_list}

If no self-organisation is done, a SOLL behaves similarly to a collection.
New elements are inserted at the end of the list, and searching is done
from the start to the end using pointer-chasing.  For efficiencies sake,
we keep pointers to both the start and the end of the list. Insertion,
therefore, has a $O(1)$ cost.

However, if self-organisation is done via the {\it move-to-front} (MFT)
policy, each time an element is accessed, it is moved to the start of the
list. This means that the next time we search for the same element, it
will be found more quickly, since it is near the start of the list. Once
the element is found, this re-ordering also has a complextity of $O(1)$.

The MFT policy can sometimes be too aggresive; uncommon elements that
are searched for will move to the front, potentially displacing elements that
are being searched for more frequently. An alternative is the
{\it transpose} policy. When an element is accessed, it is moved one
place in the list closer to the start. This once again can be done in
constant time, providing we keep a copy of a pointer to the previous
element during the search, or alternatively, use a doubly-linked list
e.g. each element has both a next and a previous pointer.


\begin{exercise}

Implement a SOLL using linked lists to implement these different policies
and store strings.



\begin{itemize}
\item {\bf $70\%$}
Write the files \verb^Linked/specific.h^ and \verb^Linked/linked.c^. Using
the \verb^soll.mak^ makefile, you can compile this against the
\verb^soll.h^ file and one of the test/driver files \verb^testsoll.c^,
\verb^build.c^ or \verb^uniq.c^.

The ADT should implement the standard functions: \verb^soll_init^,
\verb^soll_add^, \verb^soll_remove^, \verb^soll_isin^,
\verb^soll_tostring^, \verb^soll_size^ and \verb^soll_free^.  In addition,
write \verb^soll_freq()^ which reports the frequency of access for a
particular element. Therefore, each element of the list has the
overhead of keeping track of the number of times it has been accessed.


\item {\bf $20\%$}
Show a testing strategy on the above by submitting \verb^testing.txt^
where you give details of
unit testing, white/black-box testing done on your code. Describe any
test-harnesses used. Convince me that every line of your C code
has been tested.

\item {\bf $10\%$}
Show an extension to the project in a direction of
your choice via \verb^extension.txt^.
It should demonstrate your {\bf understanding} of some aspect
of programming or S/W engineering. Make sure it's clear what has been
done, why, and how to compile it.
\end{itemize}


\end{exercise}
