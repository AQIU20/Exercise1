\nsection{NLab}

\begin{itemize}
\item The programming language MATLAB (originally available in the late
1970s, for free) is one of the most widely used scientific langaues in
the world.
\item One of the most interesting things about MATLAB, is that every
single variable is stored as a $2D$ array - even a scaler integer
is simply a $1\times1$ array \footnote{Actually as the name implies,
they are all stored as matrices, but we will ignore the mathematical
interprettion here.}.
\item Here, we develop a very simple version of this concept - a language
that allows such arrays to be created or read from file, and functions performed
on each part of the array, one element at a time.
\end{itemize}

\subsection*{Examples}

\lstinputlisting[columns=fixed,basicstyle=\small\ttfamily\color{ocre},numbers=none,backgroundcolor=\color{darkgray}]{../Code/Week12/NLab/setprinta.nlb}
\lstinputlisting[columns=fixed,basicstyle=\small\ttfamily\color{ocre},numbers=none,backgroundcolor=\color{darkgray}]{../Code/Week12/NLab/setprinta.nlb}

sets the variable I to have the value $5$, and prints it to the screen:

\lstinputlisting[basicstyle=\scriptsize\ttfamily,frame=none,numbers=none]{setprinta.results}



You can create an array full of ones and add $2$ to each cell of the array:
\lstinputlisting[columns=fixed,basicstyle=\small\ttfamily\color{ocre},numbers=none,backgroundcolor=\color{darkgray}]{../Code/Week12/NLab/setprintb.nlb}

\lstinputlisting[basicstyle=\scriptsize\ttfamily,frame=none,numbers=none]{setprintb.results}


Loops are possible too, here a loop counts from $1$ to $10$ via the variable $I$ and computes factorials in the variable $F$. Both variables are scalars (a $1\times1$ array)~:
\lstinputlisting[columns=fixed,basicstyle=\small\ttfamily\color{ocre},numbers=none,backgroundcolor=\color{darkgray}]{../Code/Week12/NLab/loopa.nlb}

\lstinputlisting[basicstyle=\scriptsize\ttfamily,frame=none,numbers=none]{loopa.results}

Such loops (like in C) have counters stored in a variable. Changing this variable inside the loop can affect when the loop ends~:
\lstinputlisting[columns=fixed,basicstyle=\small\ttfamily\color{ocre},numbers=none,backgroundcolor=\color{darkgray}]{../Code/Week12/NLab/loopb.nlb}

\lstinputlisting[basicstyle=\scriptsize\ttfamily,frame=none,numbers=none]{loopb.results}

As grammar tells you, loops can be nested too~:
\lstinputlisting[columns=fixed,basicstyle=\small\ttfamily\color{ocre},numbers=none,backgroundcolor=\color{darkgray}]{../Code/Week12/NLab/nestedloop.nlb}

\lstinputlisting[basicstyle=\scriptsize\ttfamily,frame=none,numbers=none]{nestedloop.results}

\subsection*{The Formal Grammar}
\lstinputlisting[language=bash,basicstyle=\scriptsize\ttfamily,frame=none,numbers=none]{../Code/Week12/NLab/nlab.grammar}

\begin{exercise}
\begin{itemize}
\item {\bf $30\%$}
Implement a recursive descent parser - this will report
whether or not a given NLab program follows the formal grammar or not.
The input file is specified via \verb^argv[1]^ - there is {\bf no} output if
the input file is {\bf valid}. Elsewise, a non-zero \verb^exit^ is made.

\item {\bf $30\%$}
Extend the parser, so it becomes an interpreter. The instructions are
now `executed'. Do not write a new program for this, simply extend your
existing parser.

\item {\bf $20\%$}
Show a testing strategy on the above -
you should give details of
unit testing, white/black-box testing done on your code. Describe any
test-harnesses used. In addition, give examples of the output of many different
NLab programs. Convince me that every line of your C code
has been tested.

\item {\bf $20\%$}
Show an extension to the project in a direction of
your choice. It should demonstrate your {\bf understanding} of some aspect
of programming or S/W engineering. If you extend the formal grammar
make sure that you show the new, full grammar.
\end{itemize}

\subsection*{Hints}
\begin{itemize}
\item Don't try to write the entire program in one go. Try a cut
down version of the grammar first, e.g.:
\begin{verbatim}
<PROG> ::== "BEGIN" { <INSTRCLIST>
INSTRCLIST ::= "}" | <INSTR> <INSTRCLIST>
<INSTR> ::= <PRINT> | <SET>
<PRINT} ::= "PRINT" <VARNAME>
<SET> ::= <VARNAME> ":=" <POLISHLIST>
<POLISHLIST> ::= <POLISH> <POLISHLIST> | ";"
<POLISH> ::= <VARNAME> | <INTEGER>
\end{verbatim}
\item The language is simply a sequence of words (even the semi-colons),
so use \verb^fscanf()^.
\item Some issues, such as what happens if you use an undefined variable,
or if you use a variable before it is set, are not explained by the formal
grammar. Use your own common-sense, and explain what you have done.
\item Once your parser works, extend it to become an interpreter. DO NOT
aim to parse the program first and then interpret it separately. Interpreting
and parsing are inseparably bound together.
\item Start testing very early - this is a complex beast to test and trying to
do it near the end won't work.
\item In NLab, all variables are global i.e. they are not local to loops etc.
\end{itemize}

\subsection*{Submission}
Your testing strategy will be explained in \verb^testing.txt^, and your extension
as \verb^extension.txt^. For the parser, interpreter and extension sections, make
sure there's a \verb^Makefile^, so that I can easily build the code using \verb^make parse^,
\verb^make interp^ and \verb^make extension^. Submit a single \verb^nlab.zip^ file.

\end{exercise}
