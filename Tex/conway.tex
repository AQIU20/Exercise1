\nsection{Conway's Soldiers}

The one player game, {\it Conway's Soldiers} (sometimes known as {\it
Solitaire Army}), is similar to peg solitaire. For this exercise,
Conway's board is a $7$ (width) $\times$ $8$ (height) board with tiles
on it. The lower half of the board is entirely filled with tiles (pegs),
and the upper half is completely empty.  A tile can move by jumping
another tile, either horizontally or vertically (but never diagonally)
onto an empty square. The jumped tile is then removed from the board.
A few possible moves are shown below:\\

\newcommand{\A}{|[fill=ocre,text=black]|X}
\newcommand{\B}{|[fill=gray,text=black]|.}

\begin{tikzpicture}[every node/.style={anchor=base,text depth=.5ex,text height=2ex,text width=1em,outer sep=0pt,align=center,inner sep=0pt}]
\matrix [matrix of nodes,draw=white,nodes in empty cells]
{
\B&\B&\B&\B&\B&\B&\B\\
\B&\B&\B&\B&\B&\B&\B\\
\B&\B&\B&\B&\B&\B&\B\\
\B&\B&\B&\B&\B&\B&\B\\
\A&\A&\A&\A&\A&\A&\A\\
\A&\A&\A&\A&\A&\A&\A\\
\A&\A&\A&\A&\A&\A&\A\\
\A&\A&\A&\A&\A&\A&\A\\
};
\end{tikzpicture}
\hspace*{0.3in}
\begin{tikzpicture}[every node/.style={anchor=base,text depth=.5ex,text height=2ex,text width=1em,outer sep=0pt,align=center,inner sep=0pt}]
\matrix [matrix of nodes,draw=white,nodes in empty cells]
{
\B&\B&\B&\B&\B&\B&\B\\
\B&\B&\B&\B&\B&\B&\B\\
\B&\B&\B&\B&\B&\B&\B\\
\B&\B&\A&\B&\B&\B&\B\\
\A&\A&\B&\A&\A&\A&\A\\
\A&\A&\B&\A&\A&\A&\A\\
\A&\A&\A&\A&\A&\A&\A\\
\A&\A&\A&\A&\A&\A&\A\\
};
\end{tikzpicture}
\hspace*{0.3in}
\begin{tikzpicture}[every node/.style={anchor=base,text depth=.5ex,text height=2ex,text width=1em,outer sep=0pt,align=center,inner sep=0pt}]
\matrix [matrix of nodes,draw=white,nodes in empty cells]
{
\B&\B&\B&\B&\B&\B&\B\\
\B&\B&\B&\B&\B&\B&\B\\
\B&\B&\B&\B&\B&\B&\B\\
\B&\B&\A&\A&\B&\B&\B\\
\A&\A&\B&\B&\A&\A&\A\\
\A&\A&\B&\B&\A&\A&\A\\
\A&\A&\A&\A&\A&\A&\A\\
\A&\A&\A&\A&\A&\A&\A\\
};
\end{tikzpicture}
\hspace*{0.3in}
\begin{tikzpicture}[every node/.style={anchor=base,text depth=.5ex,text height=2ex,text width=1em,outer sep=0pt,align=center,inner sep=0pt}]
\matrix [matrix of nodes,draw=white,nodes in empty cells]
{
\B&\B&\B&\B&\B&\B&\B\\
\B&\B&\B&\B&\B&\B&\B\\
\B&\B&\B&\B&\B&\B&\B\\
\B&\B&\B&\B&\A&\B&\B\\
\A&\A&\B&\B&\A&\A&\A\\
\A&\A&\B&\B&\A&\A&\A\\
\A&\A&\A&\A&\A&\A&\A\\
\A&\A&\A&\A&\A&\A&\A\\
};
\end{tikzpicture}




The user enters the location of an empty square they'd like to
get a tile into, and the program demonstrates the moves that enables
the tile to reach there (or warns them it's impossible). To do this you
will use a list of boards. The initial board is put into this list.
Each board in the list is, in turn, read from the list and all possible
moves from that board added into the list. The next board is taken, and
all its resulting boards are added, and so on.

%However, one problem
%with is that repeated boards may be put into the queue and cycles
%occur.  This soon creates an explosively large number of boards
%(several million). You can solve this by only adding a board into the
%list if an identical one has never been put into the list before.  A
%linear search is acceptable for this task.
%
Each structure in the list
will contain (amongst other things) a board and a record of its parent
board, i.e. the board that it was created from.

\begin{exercise}
Write a program that:
\begin{itemize}
\item Inputs a target location for a tile to reach (x in argv[1], y in argv[2]).
\item Demonstrates the correct solution (reverse order is fine) using plain text.
\end{itemize}
\end{exercise}

Use the algorithm described above and not anything else.


